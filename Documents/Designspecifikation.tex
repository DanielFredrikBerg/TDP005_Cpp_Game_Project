%!TEX TS-program = xetex
\documentclass{TDP005mall}
\usepackage[utf8]{inputenc}
\usepackage[swedish]{babel}
\usepackage[export]{adjustbox}
\usepackage{tabularx}
\usepackage{caption}

\usepackage[style=authoryear, backend=biber]{biblatex}
\addbibresource{reference.bib}
\usepackage{csquotes}

\renewcommand*{\contentsname}{Innehållsförteckning}

\newcommand{\version}{Version 1.0}
\author{Daniel Huber, \url{danhu849@student.liu.se}\\
  Viktor Rösler, \url{vikro653@student.liu.se}}
\title{Designspecifikation}
\date{2020-11-25}
\rhead{Daniel Huber\\
Viktor Rösler}

% For aligning captions to the left.
\captionsetup{justification=raggedright,singlelinecheck=false} 

\begin{document}
\projectpage
\tableofcontents
\newpage
\section{Revisionshistorik}
\begin{table}[!h]
\begin{tabularx}{\linewidth}{|l|X|l|}
\hline
Ver. & Revisionsbeskrivning & Datum \\\hline
1.0 & Designspecifikation 1:a utkast & 201125 \\\hline
\end{tabularx}
\end{table}


\section{Klassdiagram}
% Viktigaste innehållet i designspecen är ett klassdiagram som visar den objektorienterade designen av spelet.
% Klassdiagram enligt UML som beskriver hela ert system. Diagrammet skall vara kommunikativt och beskriva hela systemets inre uppbyggnad och funktion. Klassdiagrammet ska innehålla alla relationer som finns mellan era klasser. För associationer ska det framgå vilken riktning samt multiplicitet relationerna har. 

\section{Detaljbeskrivning av klassen Player}

\section{Detaljbeskrivning av klassen Enemy}
% Detaljbeskrivning av två centrala klasser i ert spel. Den ena klassen som beskrivs ska vara den som motsvarar spelaren, den andra får ni välja själva. Detaljbeskrivningen ska innehålla följande:

%     Namn på klassen
%     Syftet med klassen
%     Vilka andra klasser som klassen har relationer till, och på vilket sätt de hänger ihop
%     En beskrivning av konstruktor(erna)
%     En beskrivning av de publika metoder som finns i klassen (get- och setmetoder kan exkluderas)
%     En beskrivning av variabler i klassen, och deras syfte

% För utkastet: Eftersom ni inte har skrivit särskilt mycket kod i projektet ännu, skriv detaljbeskrivningen utifrån frågeställningen: Hur är det tänkt att det här objektet interagerar med andra objekt i spelet? Hur kan spelaren veta när det exempelvis går in i en vägg? 



\section{Designdiskussion}% Vilka typer av spelare borde spela ert spel?
% En kort diskussion (1/2-1 sida) där ni motiverar er design och tar upp fördelar och nackdelar med den. Diskutera gärna också om det är något ni tycker är dåligt med den, och beskriv alternativa, bättre lösningar. 


\subsection{Filformat}
% Beskriv de externa filformat ni använt i ert spel, till exempel för highscore-listor eller banbeskrivningar. Detta är kanske inte relevant för alla, då man kan lösa uppgiften utan användning av externa format.


\newpage
\printbibliography

\end{document}

%%% Local Variables: 
%%% coding: utf-8
%%% mode: latex
%%% TeX-engine: xetex
%%% TeX-master: t
%%% End: 

