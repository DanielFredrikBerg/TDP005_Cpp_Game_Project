
\documentclass{TDP005mall}
\usepackage[utf8]{inputenc}
\usepackage[swedish]{babel}


\newcommand{\version}{Version 1.2}
\author{Daniel Huber, \url{danhu849@student.liu.se}\\
  Jens �hrnell, \url{jenoh242@student.liu.se}}
\title{Dokumentmall}
\date{2020-11-05}
\rhead{Daniel Huber\\
Jens �hrnell}



\begin{document}
\projectpage
\tableofcontents
\section{Revisionshistorik}
\begin{table}[!h]
\begin{tabularx}{\linewidth}{|l|X|l|}
\hline
Ver. & Revisionsbeskrivning & Datum \\\hline
1.2 & Kravspecifikation TDP005 & 201101 \\\hline
1.1 & Modifierad för att stödja xelatex och unicode & 150603 \\\hline
1.0 & Skapad för studenter att använda som mall till
kommande dokumentinlämningar & 140908 \\\hline
\end{tabularx}
\end{table}


\section{Spelide´}% Hur skriver man e med apostrof i latex..?
% Vad går spelet ut på? - Tänk: vad ska stå på Steam-sidan som säljer ert spel?
Din familj har blivit kidnappad av Ondskan! Du vet om att Ondskan håller till på toppen av berg omgivna av lava, men du vet inte vilket! Klättra upp för bergen genom att hoppa upp på plattformarna, undvik den ökande nivån av lava, döda eller undvik fienderna, samla power-ups och besegra bossen på toppen av berget! I denna bottom-top platforms scroller kan du spela ensam eller tillsammans med en vän. Klara banorna snabbare för högre highscore. Kan du rädda din familj i tid?

Spelaren/spelarna börjar på botten av berget vid varje nivås början med fullt liv och en pistol. Efter att ha hoppat upp på första platformen börjar lavan stiga och spelarfönstret börjar förflytta sig uppåt. I takt med att spelarfönstret och spelaren rör sig uppåt så uppkommer fler platformar spelaren kan hoppa upp till. Fiender kan finnas och röra sig över platformarna samt skjuta i bestämda tidsintervall eller när spelaren befinner sig på samma höjd som dem. Fiender kan också flyga i utrymmet mellan plattformarna. Spelaren kan inte röra sig utanför skärmen och kan inte hoppa igenom plattformar. Får spelaren noll liv eller rör lavan så är det game over. På plattformarna kan också power-ups finnas som aktiveras när spelaren går i samma ruta som dem. Dessa kan ge antingen permanenta eller temporära fördelar i nivån, men förs inte över till nästa nivå.


\subsection{Spelets mål}
Döda fiender medan du klättrar uppför berget och klättra upp för berget snabbt för en högre andel poäng. Besegra bossarna på toppen av varje berg och rädda din familj i tid.
\section{Målgrupp}% Vilka typer av spelare borde spela ert spel?
Vårt spel riktar sig till 
\section{Spelupplevelse}% Vad gör spelet underhållande att spela?

\section{Spelmekanik}% Hur interagerar man med spelet? Vilka kommandon? Vad driver spelet framåt?
\subsection{Storlek}
Allting i spelet upptar lika mycket plats, en spelruta. Detta gäller spelare, fiender, power-ups och enskilda bitar av kartan. Endast bossen är flera spelrutor stor.

\subsection{Förflyttning}
Spelaren styrs med hjälp av tangentbord och/eller gamepad och kan förflyttas åt höger eller vänster både på marken och i luften. Finns stege möjliggör detta att spelaren kan förflytta sig rakt upp. Spelare kan välja att hoppa åt både vänster, höger och rakt upp. Det finns bara en hopphöjd och denna är oföränderlig. Fiender är antingen stationära eller rör sig i förutbestämda mönster. De kan finnas på platformar eller i luften mellan platformarna.

\subsection{Projektiler}
Skott som avfyras från pistolen spelaren börjar med kan bara skjutas åt det håll spelaren är vänd åt, antingen höger eller vänster. Skotten flyger rakt i x-led och förflyttas ej i y-led. De exploderar när de träffar en vägg eller fiende. Flyger de utanför skärmen utan att träffa något så försvinner de. Om fiendetypen kan avfyra projektiler gör de så under regelbundna tidsintervall. Dör en fiende efter att ha avfyrat ett skott så fortsätter fiendens skott tills det träffar något eller åker utanför banan. När fiender, spelaren eller väggar träffas av skott visas skottets explosionssprite. Spelaren kan inte skada sig själv med sina egna skott, men spelare kan vid multiplayer välja om "friendly fire" ska vara på eller av.

\subsection{Power-ups}
Power-ups finns förutbestämt utlagda på plattformarna, men kan också släppas av fiender när de dör. Om de släpps av fiender eller inte beror på ett slumptal som genererats vid banans inläsning. För spelaren kommer det framstå som slumpartat, men är egentligen förutbestämt. 

\subsection{Spelarfönstret}
Spelare kan inte påverka spelarfönstrets förflyttning utan det kommer för spelaren att verka röra sig i konstant hastighet upp för berget.

Desto högre upp spelaren är i spelarfönstret desto snabbare verkar det som att spelarfönstret rör sig uppåt. Befinner sig spelaren konstant i toppen av skärmen kan spelaren snabbare klara nivån.

\section{Regler}% Vilka regler styr spelet?
\section{Visualisering}% Hur skall spelet se ut? (ha med en skiss) LOFI
\section{Kravformulering}% Finns under samma rubrik nedan.
\section{Kravuppfyllelse}% Motivering hur vår implementation uppfyller kraven i kravformuleringen.
\section{}


\section{Kravformulering}
\subsection{Minimikrav på projektet}
\begin{itemize}
\item Spelet ska simulera en 2D-värld i realtid. (Måste vara 2-dimensioner)
\item Minst 3 typer av objekt. Ska finnas flera instanser av minst två av dessa. Till exempel ett spelarobjekt och många instanser av två olika fiendeobjekt.
\item Ett beteende som måste finnas med är att figurerna (Objekten) ska röra sig över skärmen. Rörelsen kan följa mönster och/eller vara slumpmässig. Minst ett objekt utöver spelaren ska ha någon typ av rörelse.
\item Kollisionshantering ska finnas.
\item Ska vara enkelt att modifiera banor i spelet. Det ska vara enkelt att lägga till eller ändra banor i spelet. Detta kan exempelvis lösas genom att läsa in banor från en fil (lite som i Sokoban-labben i TDP002), eller genom att ha funktioner i programkoden som bygger upp en datastruktur som definierar en bana. 
\item Spelet ska upplevas som sammanhängande spel som går att spela.
\item En figur ska styras av spelaren, antingen med tangentbordet eller med musen. Du kan även göra ett spel där man spelar två stycken genom att dela på tangentbordet (varje spelare använder olika tangenter). Då styr man var sin figur.
\item Världen (spelplanen) kan antas vara lika stor som fönstret (du kan göra en större spelplan med scrollning, men det blir lite krångligare).
\item Det ska finnas kollisionshantering, det vill säga, det ska hända olika saker när objekten möter varandra, de ska påverka varandra på något sätt. T.ex kan ett av objekten tas bort, eller så kan objekten förvandlas på något sätt, eller så kan ett nytt objekt skapas. (Ett exempel på att skapa/ta bort objekt är när man i Space Invaders trycker på skjuta-knappen, t.ex en musknapp, då avfyras ett laserskott och skottet blir då en ny figur som skapas och placeras i världen, på en position vid laserkanonens mynning. Skottet rör sig framåt (uppåt) och om det träffar ett fiendeskepp tas både skottet och skeppet bort, om skottet kommer utanför spelplanen, dvs det missar, tas det endast bort.)
\end{itemize}

\subsection{Krav på koden}

\item    Koden ska följa en accepterad kodstandard. Koden ska gå att kompilera och köra i Ubuntu.
\item    Koden ska vara väldokumenterad. Doxygen ska användas.
\item    "Information hiding" ska användas (ingen åtkomst av instansvariabler utanför klassen).
\item    Designen ska vara modulär med så få beroenden mellan klasser som möjligt.
\item    Det ska finnas en gemensam basklass (Sprite) för alla figurer i programmet. Övriga sprite-figurer ska ärva från denna klass.
\item    Det ska finnas abstrakta metoder, åtminstone för er Sprite-hierarki.
\item    Main-funktionen ska vara liten, huvuddelen av spelet ska finnas i de klasser ni designar. Ett exempel på en lämplig main-funktion:

    int main {
       Game game;
       game.run();
       return 0;
    }
        

\item    Grafiken ska realiseras med hjälp av SFML.
\item    Tillsammans med koden ska en Makefile lämnas in. Handledaren ska kunna köra koden i terminalen med hjälp av denna Makefile.



\section{Rubrik 3}
Text för 3.

\subsection{Underrubrik till 3}
Text för 3.1.

\subsubsection{Underrubrik till 3.1}
Text för 3.1.1

\end{document}

%%% Local Variables:
%%% mode: latex
%%% TeX-master: t
%%% End:
