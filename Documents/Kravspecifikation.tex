\documentclass{TDP005mall}
\usepackage[utf8]{inputenc}

\usepackage[swedish]{babel}
\usepackage{tabularx}

\renewcommand*\contentsname{Innehållsförteckning}

\newcommand{\version}{Version 1.2}
\author{Daniel Huber, \url{danhu849@student.liu.se}\\
  Viktor Rösler, \url{vikro653@student.liu.se}}
\title{Kravspecifikation}
\date{2020-11-22}
\rhead{Daniel Huber\\
Viktor Rösler}



\begin{document}
\projectpage
\tableofcontents
\newpage
\section{Revisionshistorik}
\begin{table}[!h]
\begin{tabularx}{\linewidth}{|l|X|l|}
\hline
  Ver. & Revisionsbeskrivning & Datum \\\hline
1.2 & Kravspecifikation TDP005 & 201122 \\\hline
1.1 & Modifierad för att stödja xelatex och unicode & 150603 \\\hline
1.0 & Skapad för studenter att använda som mall till
kommande dokumentinlämningar & 140908 \\\hline
\end{tabularx}
\end{table}


\section{Spelid\'{e} }
% Vad går spelet ut på? - Tänk: vad ska stå på Steam-sidan som säljer ert spel?
Din familj har kidnappads av Ondskan! Du vet om att Ondskan håller till på toppen av berg omgivna av lava, men du vet inte vilket! Klättra upp för bergen genom att hoppa upp på plattformarna, undvik den ökande nivån av lava, döda eller undvik fienderna, samla power-ups och ta dig uppför berget så snabbt som möjligt! I denna bottom-top platformsscroller kan du spela ensam eller tillsammans med en vän. Kan du rädda din familj i tid?

\subsection{Spelets mål}
Varje nivå avklaras när toppen av nivån nås av spelaren och spelet avklaras när toppen på den sista nivån nås. På vägen upp mot toppen dödas eller undviks fiender. 

\section{Målgrupp}% Vilka typer av spelare borde spela ert spel?
Spelet riktas mot spelare som underhålls av spel med en ökande svårighetsgrad för varje nivå och där nivåer måste avklaras under en viss tid. 

\section{Spelupplevelse}% Vad gör spelet underhållande att spela?
Spelaren tvingas i varje nivå hitta den mest tidseffektiva vägen upp samtidigt som fiender undviks eller dödas. Vid spelets avklarande belönas spelaren med en känsla utav att ha klarat av en utmaning då spelet designats att vara svårt.

\subsection*{Multiplayer}
Multiplayer kan aktiveras för att möjliggöra lokalt spel för två spelare. Vid den ena spelarens död kan den andre spelaren fortsätta nivån. Antingen hjälps spelare åt att klara nivåerna eller så kan friendly-fire aktiveras för att öka svårighetsgraden ytterligare då spelarna skadas av den andres projektiler.

\section{Spelmekanik}% Hur interagerar man med spelet? Vilka kommandon? Vad driver spelet framåt?
Nåvåerna startas med spelarkaraktären vid botten av berget med tre liv och en pistol. Efter att första plattformen nås börjar lavan stiga. Genom nivån följs spelaren av spelarfönstret. I takt med att spelaren rör sig uppåt syns fler platformar spelaren kan hoppa upp till. På plattformar dödas eller undviks fiender av spelaren. Mellan platformar kan flygande fiender finnas. Spelarens väg upp mot toppen försvåras av att platformar inte kan hoppas igenom. Game over triggas om spelaren skadas till noll liv. I nivån kan power-ups plockas upp och ge spelaren tillfälliga fördelar eller återställa förlorade liv. 

\subsection{Förflyttning} % Lägga in tabell för xbox kontroll förflyttning.

\begin{table}[h!]
  \centering
  \caption{Tangentbordskommandon\label{tab:1}}
\begin{tabular}{|l|l|}
\hline
Tangent & Resultat \\\hline
↑, Z & Spelaren hoppar uppåt. \\\hline
← & Spelaren förflyttas åt vänster. \\\hline
→ & Spelaren förflyttas åt höger. \\\hline
↑ + ←, Z + ← & Spelaren hoppar åt vänster. \\\hline
↑ + →, Z + → & Spelaren hoppar åt höger. \\\hline
X & Ett skott avfyras i den riktningen spelaren är vänd åt. \\\hline
\end{tabular}
\end{table}


% Lägg till gamepad ifall vi bestämmer oss för det - i bör-kraven.
Spelaren styrs med hjälp av tangentbordet enligt tabell:\ref{tab:1}. I luften styrs spelaren av respektive riktningstangent. Spelare förflyttas i både vänster-, höger- och höjdled med hjälp av hopp. Det finns bara en hopphöjd och denna är oföränderlig. Vid kontinuerlig rörelse åt något håll byggs momentum upp och spelarkaraktären förflyttas ytterligare en liten bit när spelarinput slutas att ges.

\subsection{Fiender}
Över nivåerna förflyttas fiender i förutbestämda mönster. De finns på platformar eller i luften mellan platformarna. Det ges ingen visuell representation i spelet av hur många återstående liv en fiende har. Tre typer av fiender finns.

\subsubsection*{Flygande fiende 1}
\begin{figure}[h!]
  \caption{Flygande fiende 1 sprite-sheet\label{fig:1}}
  \centerline{\includegraphics[width=5cm, height=2cm]{/home/danhu849/tdp005/Documents/images/flying_enemy1.png}}
\end{figure}

\begin{table}[h!]
  \centering
  \caption{Egenskaper: Flygande fiende 1\label{tab:2}}
\begin{tabular}{|l|l|}
\hline
Attribut & Värde \\\hline
Liv & 1 \\\hline
Skada & 1 \\\hline
Speciell förmåga & Är inte bunden till plattformar. \\\hline
\end{tabular}
\end{table}
%\newpage
\subsubsection*{Gående fiende 1}
\begin{figure}[h!]
  \centerline{\includegraphics[width=5cm, height=2cm]{/home/danhu849/tdp005/Documents/images/slime_ememy.png}}
  \caption{Gående fiende 1 sprite-sheet\label{fig:2}}
\end{figure}

\begin{table}[h!]
  \centering
  \caption{Egenskaper: Gående fiende 1\label{tab:3}}
\begin{tabular}{|l|l|}
\hline
Attribut & Värde \\\hline
Liv & 3 \\\hline
Skada & 1 \\\hline
Speciell förmåga & Har mer liv. \\\hline
\end{tabular}
\end{table}

\subsubsection*{Hoppande fiende 1}
\begin{figure}[h!]
   \caption{Hoppande fiende 1 sprite-sheet\label{fig:3}}
  \centerline{\includegraphics[width=5cm, height=2cm]{/home/danhu849/tdp005/Documents/images/jumping_enemy1.png}}
\end{figure}

\begin{table}[h!]
  \centering
  \caption{Egenskaper: Hoppande fiende 1\label{tab:4}}
\begin{tabular}{|l|l|}
\hline
Attribut & Värde \\\hline
Liv & 2 \\\hline
Skada & 1 \\\hline
Speciell förmåga & Kan hoppa och skjuta projektiler. \\\hline
\end{tabular}
\end{table}

\subsection{Spelarfönstret}
Spelaren följs kontinuerligt av spelarfönstret och den nedersta delen av spelarfönstret upptas till en början av lava. Vid snabb förflyttning upp kan lavan hamna ur bild och vid för långsam förflyttning upptas en allt större del av spelarskärmen av lava tills dess att spelaren träffas och game over visas på skärmen.


\subsection{Spelmeny}
Det första spelaren möts av efter programmets start är spelets meny i form av valbara knappar. Alternativen; nivåer, inställningar för tangentkommandon samt preferenser över ljud och musik manövreras över med piltangenterna och väljs med mellanslag. \ref{fig:4} Menyerna ritas ut framför spelet som ritats ut i bakgrunden.

\section{Regler för spelet}% Vilka regler styr spelet?

\subsection{Single player}
Nivån startas med spelaren vid max antal liv, tre. Dessa representeras av en health-bar i övre delen av spelarfönstret. Vid fullt liv kan extra liv plockas upp, men extra liv utöver max ges inte. Förloras alla liv visas game over i spelarfönstret. Vid kontakt med fiende skadas spelaren ett liv.

\subsection{Multiplayer}
Nivån startas med båda spelarna vid fullt liv. Första spelarens liv representeras i översta vänstra hörnet och andra spelarens i det högra. Spelarna skiljs åt av färger. Spelare ett ges färgen gul, spelare två färgen grön. I nivån upptas samma utrymme omöjligen av båda spelkaraktärer samtidigt. En spelare kan flyttas av den andre om denne inte ger någon input via tangentbordet eller gamecontrollern.

\subsection{Fiender}
Vi fienders död upplöses de. Spelaren skadas ett liv vid kontakt med fiende. Fiender kan ej uppta samma utrymme på skärmen.

\subsection{Projektiler}
Skott som avfyras med spelarens vapen kan bara skjutas i den riktning spelarkaraktären är vänd åt, antingen höger eller vänster. Projektiler kan inte avfyras uppåt. Skotten färdas alltid i samma höjdled de sköts ifrån. Om projektiler avfyras i luften färdas projektilen i samma y-led till kollidering med spelbanan eller spelfönstrets kant. Träffas fiender av spelarens projektiler åsamkas ett i skada på fienden.

\subsection{Kollisionshantering} % Lägga till rätt korrelation mellan labels nedan och refs i Ska- och bör-kraven nedan.
Kollision definieras av att en entitets hitbox nivåkoordinater delvis eller helt delas med en annan entitets hitbox nivåkoordinater.
\subsubsection*{Kollision mellan fiender eller fienders projektiler och spelaren.\label{}}
Ett hp av spelarens liv förloras och spelaren ges odödlighet i 1.618 sekunder. Projektilen tas sedan bort och explosionsanimering visas vid projektilens kollionskoordinater.

\subsubsection*{Kollision mellan spelarens projektil och fiende.\label{}}
Projektilens explosionsanimering spelas upp vid projektilens kollisionskoordinater. Ett av fiendens liv förloras. Finns inga liv kvar hos fienden tas fienden bort från spelarfönstret. Finns liv kvar så fortsätts fiendens rörelsemönster.

\subsubsection*{Kollision mellan en fiende och en annan fiende.\label{}}
Riktningen på båda fienders banor ändras till motsatt riktning.

\subsubsection*{Kollision mellan en projektil och en annan projektil.\label{}}
Projektilernas bana fortsätts utan att något speciellt händer.

\subsubsection*{Kollision mellan spelaren och lavan.\label{}}
Alla spelarens liv förloras, texten "game over" samt en knapp om att starta om nivån och en knapp till nivåmenyn visas på spelskärmen.

\subsubsection*{Kollision mellan projektil och vägg eller plattform.\label{}}
En explosionsanimering visas vid projektilens kollisionskoordinater. Sedan tas projektilen bort.

\newpage
\section{Visualisering}% Hur skall spelet se ut? (ha med en skiss) LOFI. Behöver en när spelaren hoppar. En när spelaren rör lavan. En vid meny val exempel. En när spelaren skjuter.

\begin{figure}[h!]
  \caption{LOFI-exempel Nivå start.\label{fig:4}}
  \centerline{\includegraphics[width=\textwidth, height=15cm]{/home/danhu849/Pictures/game_example.png}}  
\end{figure}

I övre högra hörnet representeras spelarens återstående liv av gröna kuber i en health-bar. Varje nivå \ref{fig:4} startas med spelkaraktären i nedre delen av spelarfönstret alldeles ovanför lavan. En eller flera utstakade vägar av plattformar i olika höjd placeras över nivån.

\subsection{Ska-krav på projektet} % Lägg till referenser till vart respektive krav diskuteras utförligare.
\begin{enumerate}
\item Två spelare ska kunna spela spelet samtidigt. Spelare ett ges gul färg, spelare två ges grön färg.
\item Spelet ska kunna spelas med både handkontroll \ref{tab:} och tangentbord \ref{tab:1}. 
\item Spelet ska ha en nivåmeny med möjlighet att välja spelnivå att spela.
\item Efter avklarad nivå kan spelaren tas till nästa nivå eller nivåmenyn med hjälp av knappar. 
\item Spelets nivåer ska vara vertikalt större än spelfönstret.
\item Spelaren ska kunna röra en pixelerad figur i alla vädersträck med tangentbordet \ref{tab:1} eller handkontroll \ref{tab:2}.
\item Spelaren ska kunna stå och hoppa på plattformar.
\item Spelarens gång och hopp ska animeras med hjälp av ett sprite-sheet.
\item Vid projektilers kollisionskoordinater med spelare, fiender eller plattformar ska projektilens explosionsanimering visas.
\item Spelaren ska efter första hoppet börja bli jagad av stigande lava. Nudas lavan av spelaren ska spelaren dö.
\item Nivån ska avslutas när toppen nås av spelaren.
\item Spelet ska kunna pausas och återupptas genom att trycka på tangenten p.
\item Vid spelarens död ska en Game Over vy med möjlighet att starta om banan visas i spelfönstret.
\item Alla spelfigurer förutom de flygande fienderna\ref{fig:1} påverkas av gravitation som drar dem nedåt mot spelarfönstrets botten.
\item Projektiler ska kunna avfyras mot den sida spelaren är vänd åt från spelarens position.
\item Projektilerna ska vid kollision med fiender eller väggar upplösas.
\item Vid kollision mellan projektil och, vägg, spelare eller fiende ska projektilens explosionsanimering spelas upp.
\item Tre olika typer av fiendeobjekt som har olika utseende ska finnas. En flygande typ \ref{fig:1}, en hoppande typ \ref{fig:2} och en gående typ\ref{fig:3}.
\item Det ska finnas en tydligt synlig livmätare i spelarfönstret där spelarens återstående liv visas. \ref{fig:4}
\item Spelet ska kunna spelas på skolans datorer.
\end{enumerate}


\subsection{Bör-krav på projektet}
\begin{enumerate}
\item Spelare ska vid multiplayer kunna välja om "friendly fire" ska vara på eller av i spelets inställningar. 
\item Fiender ska kunna avfyra projektiler som färdas i rak bana. *
\item I slutet på banan ska det finnas en boss. I bossfighten stoppas lavan vid spelarfönstrets botten.
\item Ljudeffekt ska spelas upp när spelaren skadas.
\item Ljudeffekt ska spelas upp när spelarens projektiler avfyras.
\item Ljudeffekt ska spelas upp när spelare samt fiender dödas.
\item Spelet ska ha en meny för inställningar där musik kan sättas av och på.
\item Spelet ska ha en meny för inställningar där friendly-fire kan sättas av och på.
\item Spelet ska ha en meny för inställningar där tangentbordskontroller \ref{tab:1} kan ändras.
\item Spelet ska spela bakgrundsmusik.
\end{enumerate}

\subsection{Krav på koden} % Skrivit dem från funktionell synvinkel.
\begin{enumerate}
\item    Spelkoden ska gå att kompilera och köra i Ubuntu 20.04.
\item    Koden ackompanjeras med dokumentation.
\item    Det ska ej gå att ändra spelkoden från spelets gränssnitt.
\item    Grafiken ska realiseras med hjälp av SFML.
\item    Koden ska kompileras och köras med en Makefile.
\end{enumerate}



\end{document}

%%% Local Variables:
%%% mode: latex
%%% TeX-master: t
%%% End:
