\documentclass{mall}

\newcommand{\version}{Version 1.1}
\author{Daniel Huber, \url{danhu849@student.liu.se}\\
  Viktor Rösler, \url{vikro653@student.liu.se}\\
  }
\title{Gruppkontrakt:\\ Daniel Huber och Viktor Rösler}
\date{2020-11-14}
\rhead{Daniel Huber, danhu849 \\
Viktor Rösler, vikro653}


\begin{document}
\projectpage

\section{Revisionshistorik}
\begin{table}[!h]
\begin{tabularx}{\linewidth}{|l|X|l|}
\hline
Ver. & Revisionsbeskrivning & Datum \\\hline
1.1 & Gruppkontrakt Daniel Huber och Viktor Rösler & 201114 \\\hline
1.0 & Gruppkontrakt -- Exempel & 201001 \\\hline
\end{tabularx}
\end{table}

\section{Förutsättningar}
\label{prereq}

\begin{itemize}
\item \textbf{Daniel Hubers önskelista för optimerad egeninsats i grupparbete:}

  \begin{itemize}
  \item Min dygnsrytm är från mellan 4 och 5 till 19 och 20. Effektiv jobbtid med lunch mellan 06 och 18.
  \item Får absolut inte bli störd när jag sover.
  \item Måste få tillfälle att motionera 1-1.5 timmar per dag.
  \item Jobbar bäst i en miljö där jag bara hör min datorfläkt och mina knapptryckningar på tangentbordet.
  \item Måste följa mitt mat- och sovschema för att inte riskera att bli trött och grinig.
  \item Arbetar gärna på helger om gruppmedlem gör detsamma. 
  \item Medan jag arbetar behöver jag ta små pauser ungefär varannan timme.
  \item Måste ha goda marginaler till deadline och behöver en planering.
  \item Är bra på att planera, skriva planeringen och framförallt följa den.
  \item Förväntar mig att gruppen också följer planeringen, men meddelar exakt då något försenas.
  \item När jag känner att någon tagit för lång tid med ett arbete och inte kommer hinna till deadline utan att säga till tenderar jag att ta över uppgiften och köra över alla andras åsikter bara för att möta deadline.
  \item Inte rädd att lägga ner mycket tid på arbetet, men förväntar mig då samma av gruppmedlemmar.
  \item Vill ha kritik och feedback direkt.
  \item Förväntar mig att bli konfronterad vid oklarheter under arbetetsgång gällande arbetet jag utför eller planeringen.
  \item Blir mer driven av komplimanger om mitt utförda arbete.
  \item Jag tycker inte om att inte få kredibilitet när andra använder mitt arbete.
  \item Behöver ha någon form av möte/kontroll/snack om arbetets progression sen dagen innan och vad som planerats för kommande dag. 
  \item Vill att det jag producerar ska hålla hög kvalitet, men inser vikten av att först uppfylla de grundläggande kraven och får inte ångest över att sänka mina förväntningar för att hinna lämna in arbete innan deadline. 
  \item Tycker det inte är värst kul att skriva dokument, men jag är bra på det.
  \item Frågar alltid gruppmedlemmar om hjälp när jag inte vet något och känner att jag inte kan ta reda på det själv inom rimlig tid.
  \item Svarar gärna på frågor och förklarar gärna programmeringskoncept.
  \item Är bra på att snabbt ta in information från stora texter och tycker jag har lätt för att sortera även större informationskällor.
  \item Nås lättast via telefon: 072-565 35 22 eller Discord.
  \end{itemize}

\item \textbf{Viktor Röslers önskelista för optimerad egeninsats i grupparbete:}
  \begin{itemize}
  \item Min dygnsrytm är från mellan 7 och 8 till 23 och 00. Jag behöver egentid från 17 till 19 för att motionera och äta.  
  \item Jag tycker det är viktigt att man kan ta ledigt på kvällar och helger om man vill, men jag planerar att arbeta på projektet även då.  
  \item Mitt främsta mål med projektet är att bli bättre på mjukvaruutveckling. Mål om högre betyg anpassar jag efter gruppens önskemål. 
  \item Jag kommer att vara ute i god tid med att utföra arbetsuppgifter och möta dealines. Om andra gruppmedlemmar vill vänta kommer jag att arbeta i förväg på egen hand.
  \item Behöver ha någon form av frekventa möten/kontroller/snack om arbetets progression och vad som ska göras näst.
  \item Jag vill att arbetsuppgifter som ingen gruppmeddlem tycker är så värst kul delas upp, och att arbetsuppgifter vi tycker är kul görs tillsammans. 
  \item Det är viktigt för mig att alla gruppmedlemmar blir hörda och får möjlighet att påverka.
  \item Gillar att lägga extra tid på programmeringen för att höja kvaliteten på produkten.   
  \item Är villig att hjälpa andra gruppmedlemmar med att felsöka deras kod.
  \item Diskuterar gärna alternativa lösningar, och är öppen för förslag om hur projektet utförs på bästa sätt.
  \item Om jag anser att det finns ett bättre sätt att göra något på kommer jag att säga det, och jag önskar att andra gruppmedlemmar förmedlar liknande kritik till mig.
  \item Nås lättast via skriftlig kommunikation över Discord.
  \end{itemize}

\item \textbf{}

\item \textbf{Hur ska jag bete mig för att stötta min/mina kollegor utifrån sina förutsättningar?}

 \emph{ Insikt: Här handlar det om att vara villiga att ge och ta inom gruppen. Alla kanske inte kan få som de vill
  hela tiden, men om man tänker igenom saker kan man komma fram till någonting som fungerar i stor utsträckning
  för alla inblandade.}

  Vi kommer hålla oss till ett schema för när vi ska arbeta och när vi tar raster så att vi kan respektera
  tiderna som vi kan arbeta på. Eftersom jag inte kan arbeta direkt på morgonen och min partner inte kan
  arbeta på kvällen kommer vi att arbeta tillsamman mellan ca klockan 11 och 16 på dagarna. Sedan kommer
  jag arbeta vidare efteråt och han kommer börja innan. Om jag vill kontakta min partner sent på kvällen
  så accepterar jag att hen inte kommer svara innan nästa dag.

  Eftersom min partner känner att hen vill ha väldigt konkreta uppgifter så kommer vi tillsammans göra tydliga
  specifikationer för hens uppgifter. Jag gillar att jobba med lite mer frihet men eftersom vi tydligt specificerat
  min partners uppgifter riskerar vi inte att göra samma sak. Eftersom jag är duktig på att programmera hemsidor men
  min partner gärna vill lära sig göra hemsidor kommer vi göra så att min partner får skriva utkastet till den delen,
  sedan har vi code review där vi tillsammans kan titta på hur den koden blir bättre.


\end{itemize}

\section{Hur vi arbetar tillsammans}

\begin{itemize}
\item \textbf{Vilka tider arbetar vi, och vilka tider är vi nåbara utöver detta?}

  Vi har kommit överens om att arbeta vardagar mellan vaken tid och läggdags. Då vi har olika dygnsrytmer skickar vi ett discord meddelande när vi börjar jobba om vad vi jobbar med för att undvika dubbelarbete. En gång varje dag har vi ett möte där det diskuterats vad vi gjort och vad vi kommer att jobba med de kommande 24h.

  De dagar det är skolaktiviteter träffas vi en kort stund efter skolaktiviteten för ett snabbt möte. Vissa saker förmedlas bäst ansikte mot ansikte än över discord.

\item \textbf{Hur kommunicerar vi med varandra? Vilka verktyg/kanaler använder vi? Hur och när är det okej att vi avbryter varandra?}

  Vi kommunicerar huvudsakligen huvudsakligen via discord. Meddelanden besvaras när det finns tid. Detta för att optimera flow. Vi har bestämt att det alltid bör finnas en annan åtagen uppgift att jobba med ifall en uppgift visar sig alltför svår att lösa själv och det tar lång tid att få svar.  Medlemmarna i gruppen anser sig självständiga och driftiga och litar på att den andre nyttjat de vanliga felsökningsvägarna innan förfrågan om hjälp skickas. När ett problem lösts meddelas det till den andre via discord.


  %\item \textbf{Hur gör jag för att ge min/mina kollegor möjlighet att framföra sina åsikter och tankar om uppgifter och idéer till arbetet?} ändrade detta för att vara i vi-form istället för jag
  \item \textbf{Hur gör vi för att ge varandra möjlighet att framföra åsikter och tankar om uppgifter och idéer till arbetet?}

Enligt önskemålen ovan matchar vi varandra i att vi båda vill ha kritik och feedback direkt när den andre får syn på något felaktigt eller konstigt i kod eller text. Båda har intygat att diskussioner gällande ens arbeten är välkomna och har inga problem med att radera eller skriva om kod och text ifall det skulle visa sig göra slutprodukten bättre. 

\item \textbf{Arbetar vi tillsammans med uppgifter, eller var för sig?}

  Daniel vill ha en femma i projektet och Viktor har gått med på att matcha det. Daniel anser sig bra på att skriva dokument då samtliga dokument i förra projektkursen fick betyget VG och har tagit sig an att ansvara för kravspecifikationen. Efter att kravspecifikationen fått ett VG delas resten av dokumenten upp vid ett senare tillfälle då vi bättre vet vad dokumenten ska innehålla samt vilka av de delar vi känner oss bäst att skriva.


\item \textbf{Fördelade ansvarsområden}
  
  Viktors huvudsakliga mål är att bli bättre på att koda och är väldigt duktig på det. För att optimera chansen om en femma i kursen bör han få koda så mycket som möjligt. Viktor har översiktligt ansvar för koden och ser till att den är rättad innan inlämning medan Daniel har översiktligt ansvar för dokumenten. Daniel har fått uppgiften att hålla reda på allt som behöver göras i projektet, ha koll på hur gruppen ligger till i schemat och se till att saker är klara i god tid innan deadline. 

\end{itemize}

\section{Om jag tycker att något inte fungerar}

\begin{itemize}
\item \textbf{Vad gör vi om någon kommer sent?}

  \emph{Insikt: Om du kommer 10 minuter för sent till ett möte med 3 personer har du just slösat bort 30
  minuters arbetstid (om de väntar på dig) eller den tiden det tar för de andra att förklara vad de
  redan pratat om.}

  Vi börjar utan den som är sen. En person utses att efter mötet se till att den som kom sent tar
  del av det hen missade. Den som kom sent tar med fika nästa möte.
  
  Vi börjar när alla kommit. Den som kom sist utses att protokollföra allt vi kommer fram till och
  se till att alla godkänner protokollet.

\item \textbf{Vad gör vi om någon inte slutför sina uppgifter?}

  \emph{Insikt: Om jag inte gör mina uppgifter drabbas alla andra i gruppen. Våga säga nej till uppgifter
  du inte kommer göra och våga be om hjälp för att komma igång med uppgifter som är nya för dig.}
  
  \emph{Tips: Ta upp konsekvenserna för dig/gruppen utan att beskylla. Be gruppen om hjälp leta efter
  lösningar för dig.}
  
  Vi försöker ta reda på vad som hindrar personen och hjälpa hen komma igång eller få andra
  uppgifter som inte har samma hinder.

\item \textbf{Vad gör vi om arbetsfördelningen blir ojämn?}

 \emph{Insikt: Jag tycker grupparbetet är superkul och skriver klart större delen över helgen inklusive
  några delar som inte krävs. På måndagen ligger alla andra en arbetsvecka efter. Jag tycker jag
  gjort mitt och vill inte göra mer. Alla andra är otacksamma och sura för att de nu är ensamma med
  alla tråkiga uppgifter som är kvar och riskerar till och med underkänt för det inte finns
  tillräckligt kvar för dem att göra för att bli godkända. Vad har jag gjort för fel?}

 Vi försöker ta reda på orsaken och omfördelar arbete. Om det blir ett fortsatt problem kan vi tydligare
 strukturera upp arbetet. Om det inte är möjligt för att någon i gruppen inte kan bära arbetsbördan som
 krävs för att nå betyget vi vill ha på projektet behöver vi kontakta kursledningen. Om någon vill göra
 mer än sin del eller jobba i en högre takt än vi kommit överens om vill vi att det diskuteras i förväg
 så alla får en chans att vara delaktiga.

\item \textbf{Hur tar vi upp ett problem med berörda personer?}

  \emph{Insikt: Var försiktig. Det är jobbigt att bli anklagad för ett problem. Ju fler personer som är
  med desto jobbigare blir det för den som upplever anklagelsen. Skapar ni en för jobbig situation
  riskerar ni att helt förlora en gruppmedlem. Börja istället med att närmaste kompisen på tu man
  hand försiktigt frågar hur den berörda upplever situationen och hur hen skulle vilja lösa
  situationen. Ibland visar det sig att hen inte ens uppfattat att det finns ett problem men förstår
  när hen får förklarat vad de andra upplever.}
  
 \emph{Praktiskt tips: (Privat konversation mellan dig och din kompis) Du, jag har lagt märke till att du
  ofta kommer för sent eller helt uteblir från gruppmöten, hur mår du egentligen? Är du sur på oss?}

 Den/de som känner att det är ett problem har ansvaret att ta upp det. Problemet skall i största mån tas
 upp i god tid och på ett sätt som inte pekar ut någon på ett taskigt sätt. Vi vill inte trycka ner varandra,
 men går inte och knyter handen i fickan.

 Om lösningen inte fungerar (samma problem uppstår en andra gång) tar de drabbade kontakt med kursledningen för hjälp.

\item \textbf{Hur ger jag kritik och beröm till andra personer i gruppen?}

  \emph{Insikt: Kritisera inte i onödan. Om en uppgift någon annan gjort är ''good enough'' utan att du behövt lägga tid på
  den är det ju bara positivt för dig. Vill du ändå lägga din dyrbara tid på att tipsa om
  förbättringar eller göra förändring, glöm inte uttrycka din tacksamhet för att uppgiften är löst, och skyll inte den
  extra tid du lägger på tips och ''onödigt finlir'' på någon annan än dig själv.  Omvänt: Om din
  kompis bjuder på sin tid för att tipsa dig om hur du kan lösa en uppgift ännu lite bättre så är
  det väl bra för din framtida karriär?}
  
  \emph{Praktiskt tips: (Konstruktiv kritik) Det blir väldigt jobbigt för mig att bygga vidare på den här lösningen. Om du istället gör på det
    här sättet så får vi en lösning som är lättare att arbeta med.}
  
  \emph{Insikt: Ett tips för att framföra kritik är att börja med din egen upplevelse av situationen. Fokusera helt och hållet på dig utan att skuldbelägga. Vad är konsekvenserna för dig och hur påverkar det ditt mående? Detta leder ofta till en mer konstruktiv diskussion om hur problem kan lösas.}

  \emph{Praktiskt tips: (Dialog mellan dig och kollega) När jag får vänta på implementationen av klassen PowerUp blir jag oerhört stressad, eftersom mina uppgifter till stor del beror på att klassen PowerUp är klar, och jag kan inte arbeta på kvällarna i den här veckan...}

  \emph{Insikt: Gör dig inte till mer eller mindre än du faktiskt är. Det ökar bara klyftan till de andra i gruppen. Om du får beröm och slår bort det med ''det gjorde jag på rasten'' eller liknande så blir det inte roligt att ge dig beröm, och det kan få avsändaren av berömmet att känna sig underlägsen.}
  
  \emph{Praktiskt tips: (Dialog mellan dig och kollega) \\
    (Kollega) Snyggt jobbat med klassen PowerUp!\\
    (Du) Tack, det värmer att du säger det.
  }

  Beröm ges till berörda när tillfälle ges. Vi tar emot beröm uppriktigt och förringar inte det beröm
  som ges till oss. Vi försöker framföra kritik konstruktivt och privat så att ingen behöver känna att
  de blir hackade på inför gruppen. Vi tänker igenom kritiken ordentligt på förhand och undviker att vara
  anklagande eller att attackera varandra.

\end{itemize}

\section{Utvärdering}

\begin{itemize}
\item \textbf{När ska vi påminna oss om gruppkontraktet och utvärdera hur det fungerat?}

  \emph{Insikt: Gruppkontraktet ska vara ett stöd för arbetet i kursen/projektet. Om det finns saker som
    inte fungerar i gruppen behöver det kanske omarbetas. Det kanske finns problem med att det existerande
    kontraktet inte efterlevs av alla deltagande. Kontraktet kan aldrig vara heltäckande och måste stödjas
    av en vilja att ha ett gott samarbete inom gruppen}
  
  Vi kommer 2 gånger under kursen på våra vanliga möten ta en liten stund och titta på gruppkontraktet
  och se var vi lyckats följa det och var vi misslyckats följa det. Om vi hittar något som inte fungerat
  eller som saknats i kontraktet så diskuterar vi det och om det behövs uppdaterar vi kontraktet.

  %% Vid utsatt tid: utvärdera hur gruppkontraktet har följts, fundera på ifall något i kontraktet
  %% behöver ändras, eller om något nytt behöver läggas till.

\end{itemize}

\end{document}
